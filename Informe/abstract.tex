
\textbf{Resumen} Quepy es una librería para construir sistema de respuesta a preguntas sobre datos enlazados. Al utilizar patrones estáticos para reconocer las preguntas alcanzar una gran cobertura es muy costoso, dada la gran variabilidad del lenguaje. Para solucionar este problema proponemos utilizar un clasificador bayesiano ingenuo para clasificar reformulaciones de preguntas semánticamente equivalentes y ligarlas a una misma interpretación. La falta de un corpus etiquetado para esta tarea y de trabajos anteriores con esta configuración nos llevó a utilizar un enfoque de aprendizaje activo sobre instancias y sobre características siguiendo a \citet{dualist}. Utilizaremos una novedosa representación de las instancias que incluye las concordancias con los patrones parciales del sistema, esperando capturar de esta forma la información semántica de la pregunta. En este escenario contamos con una gran cantidad de preguntas que no son reconocidas por el sistema que agrupamos es una clase \textit{other}, y muchas clases minoritarias con pocos ejemplos cada una. Los resultados indican que balancear el corpus de entrenamiento utilizado por el clasificador evita que la clase mayoritaria se convierta en la única clase reconocida por el clasificador, mientras que el entrenamiento con características aumentó en gran medida el reconocimiento de las clases minoritarias.

\textbf{Palabras claves} Aprendizaje Activo, Sistema de Respuesta a Preguntas, Datos Enlazados, Procesamiento del Lenguaje Natural.

\vspace{5 mm}

\textit{\textbf{Abstract} Quepy is a library used to build question answering systems over linked data. By using static patterns to recognize the questions the achievement of a high coverage is very expensive, given the variability of language. To solve this problem we propose to use a Naïve Bayes classifier to classify semantically equivalent reformulations of the questions and link them to a single interpretation. The lack of a tagged corpus for this tasks and previous work with this configuration led us to use an active learning approach on instances and features, following \citet{dualist}. We present a novel representation of the instances including their matches to partial patterns of the system. We hope to capture this way the semantic information of the questions.  In this scenario we have a lot of questions that are not recognized by the system grouped is a class ``other'', and many smaller classes with few examples each.  Results indicate that balancing the training corpus prevents the classifier from recognizing only one class, while training with features greatly increased the recognition of minority classes.}

\textit{\textbf{Keywords} Active Learning, Question Answering, Linked Data, Natural Language Processing.}
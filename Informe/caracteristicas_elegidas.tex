\chapter{Representación de las preguntas}

Una importante parte de cualquier problema que aborde el lenguaje natural es la representación del mismo. Las características relevantes son propias de cada problema, si bien existen numerosos ejemplos bibliográficos a tomar como modelo. El siguiente paso es identificar qué características de la pregunta son indicativas de su clase semántica, lo cual no tiene una respuesta intuitiva.

Nuestra primera aproximación fue utilizar criterios estándares en la catergorización de texto como los lemmas de las palabras y sus etiquetas POS. \citet{Sebastiani-text-categorization} describe que a pesar de su simpleza son representaciones muy poderosas y que otras más complejas no necesariamente llevarán a un mejor desempeño del clasificador.

\section{Subpatrones}

Como presentamos en el ejemplo \ref{preguntas-similares} algunas preguntas tienen tanto lemmas como etiquetas POS muy similares y sin embargo pertencen a clases distintas. La forma en la que Quepy distingue estos casos es utilizando la estructura de la frase, representada en sus patrones. Por eso, decidimos utilizar además como característica los patrones que concuerdan con la pregunta.

Esta representación por sí sola tampoco mejora nuestra situación inicial, ya que sólo reconoce las preguntas que corresponden a un patrón. La solución que encontramos para este problema fue dividir cada patrón en todos sus posibles subpatrones y asignar a cada instancias la concordancia con los subpatrones.

\begin{example} Subpatrones del ejemplo \ref{regex}.
\begin{enumerate}
\item \begin{lstlisting}
Lemma("what") + Lemma("be") + Question(Pos("DT"))
    + Group(Pos("NN"), "target") + Question(Pos("."))
\end{lstlisting}

\item
\begin{lstlisting}
Lemma("what") + Lemma("be") + Question(Pos("DT"))
+ Group(Pos("NN"), "target")
\end{lstlisting}

\item
\begin{lstlisting}
Lemma("what") + Lemma("be") + Question(Pos("DT"))
\end{lstlisting}

\item
\begin{lstlisting}
Lemma("what") + Lemma("be")
\end{lstlisting}

\item
\begin{lstlisting}
Lemma("what")
\end{lstlisting}

\item
\begin{lstlisting}
Lemma("be")
\end{lstlisting}

\item
\begin{lstlisting}
Lemma("be") + Question(Pos("DT"))
+ Group(Pos("NN"), "target") + Question(Pos("."))
\end{lstlisting}

\item
\begin{lstlisting}
...
\end{lstlisting}
\end{enumerate}

\end{example}

\section{Nombres y tipos de entidades}

Existe un tipo más de característica que determina fuertemente la semántica de la pregunta. La forma de acceder a una propiedad de una entidad dentro de una base de datos está íntimamente ligada a la estructura que le dió el usuario al ingresar los datos. Por lo tanto, la misma propiedad como ``fecha de creación'' puede tener dos nombres distintos para tipos de entidades. Por lo tanto, dependiendo de un tipo o de otro la consulta que debe generarse es dinstinta, y por eso son necesarias dos clases semánticas distintas.

\begin{example} Ejemplos con semántica diferenciada por el tipo de la entidad.
    \begin{enumerate}
        \item ``Who are the actors of Titanic?''
        \item ``Who are the actors of Friends?''
    \end{enumerate}
    Para obtener los actores que trabajaron en una película, como en el caso de la primera pregunta, debe utilizarse la relación ``/film/film/starring'', mientras que en el caso de una serie televisiva se utiliza ``/tv/tv\_program/regular\_cast''.
\end{example}

El único indicio de la clase semántica en estos casos es la entidad referenciada. Por ello, la incluímos como una característica más. Adicionalmente, agregaremos los tipos de dicha entidad en la base de conocimiento, particularmente para esta aplicación FreeBase. Cabe destacar que no todas las preguntas tienen una entidad, y en el caso de que sí tenga no siempre podemos reconocerla. Esto depende del sistema externo de reconocimiento de nombres de entidades o NER por sus siglas en inglés.

En resumen, las características propuestas para el sistema son:
\begin{itemize}
    \item Etiquetas POS.
    \item Lemmas.
    \item Concordancias a patrones parciales.
    \item Nombre de la entidad referenciada.
    \item Tipos de la entidad referenciada.
    \item Bigramas y trigramas.
    \item Bigramas mezclando Lemmas y POS.
\end{itemize}


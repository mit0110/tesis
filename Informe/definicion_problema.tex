
\chapter{Definición formal del problema}

Tanto el problema que planteamos abordar como la solución propuesta son complejos de definir ya que incluye numerosos conceptos del procesamiento de lenguaje natural. En la primera parte de esta sección definiremos un marco teórico para cada aspecto no central del problema. A partir de esta base, en la segunda parte daremos la definición formal propiamente dicha, seguida por una defición de la solución.

\section{Datos Enlazados y Sistemas de Respuesta}

\section{Quepy}

Como se mencionó anteriormente, Quepy es un marco de trabajo para crear aplicaciones de respuesta a preguntas. Una aplicación tiene dos secciones principales:
\begin{description}
    \item[Templates] Contiene las plantillas definidas por el creador de la aplicación. Cada plantilla es una expresión regular que combina distintos tipos de caracteríticas como etiquetas POS y lemas, lo que permite al sistema identificar la semántica de la pregunta únicamente en base a su sintáxis. Ante una concordancia de una pregunta ante una plantilla, Quepy genera una consulta en el lenguaje configurado agregando la información de la entidad a la cual se hace referencia. Por lo tanto, cada plantilla está asociada a una interpretación fija. Los lenguajes soportados hasta el momento son MQL y SPARQL; ambos permiten consultas posteriores a FreeBase y DBPedia.
    \item[DSL]
\end{description}

% Agregar más datos con grafiquitos sobre lo que hace quepy.

Además de ello, al ser expresiones regulares estos patrones no tienen flexibilidad y dependen fuertemente del analizador sintáctico y POS tagger que utilicen. 
%Nuestra propuesta es aplicar un clasificador automático sobre las preguntas donde cada clase es una interpretación de Quepy. De esta forma, podemos ligar muchas más reformulaciones de la misma pregunta a su correspondiente semántica y lograr mayor versatilidad para el sistema.

% Esto va acá o dentro de la parte de marco teórico?
Este enfoque de encontrar reformulaciones de una misma pregunta está enmarcado dentro del reconocimiento de implicaciones textuales y ha sido utilizado previamente para sistema de respuesta a preguntas del usuario. \citet{ou_entailement} utilizan esta técnica tomando como base preguntas modelo construidas automáticamente desde la ontología, y se centran también en la composición de patrones simples para formar otros más complejos. Sin embargo, se limitan a un dominio muy restringido que permite formar texto en lenguaje natural desde las relaciones formales entre las entidades, lo cual sería dificultoso en ontologías complejas como FreeBase. \citet{rui_relations} explican otros posibles usos de identificar estas relaciones entre las preguntas para sugerencia de preguntas relacionadas o útiles para el usuario.

%Una gran parte de los experimentos a realizar consistirá en medir el beneficio de esta nueva representación con respecto a las que comunmente se utilizan para tareas de clasificación de texto. Es lógico pensar entonces en utilizar técnicas desarrolladas para la selección e ingeniería de características.

\section{Dualist}

\section{Formalización del problema}

\section{Solución propuesta}

\subsection{Representación del problema}
